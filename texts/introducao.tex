\xchapter{Introdução}{}
\label{introducao}

Dirigir um veículo pode ser considerado uma tarefa complexa, já que envolve diversas pequenas ações coordenadas
que dependem tanto de fatores externos quanto internos. Um motorista deve saber adaptar-se a diferentes situações,
agir apropriadamente perante cada uma delas e tomar medidas rápidas em caso de emergência.

Neste contexto, a interrupção inoportuna é um perigo que pode ser fatal, pois demanda atenção do motorista em um momento
onde a distração pode levar a um acidente. Um fato que corrobora isto é o de que pessoas levam até 27\% mais tempo para
concluir uma tarefa quando são interrompidas e tendem a cometer o dobro de erros \cite{bailey2006need}. Exemplos de
interrupções que podem ocorrer para motoristas são barulhos repentinos, conversas com passageiros e notificações em
seu celular.

Outra informação que sustenta esta hipótese é a de que ao demandar a atenção dos motoristas durante a execução de
tarefas complexas sua distração também aumenta, pois a quantidade de informação que um motorista pode assimilar
depende da complexidade de sua tarefa no momento \cite{schneegass2013data}.

Um dos meios pelos quais os motoristas recebem interrupções são via notificações em dispositivos móveis.
Diariamente, pessoas que utilizam qualquer tipo de dispositivo móvel recebem dezenas de notificações
\cite{pielot2014situ}. Tais notificações demandam a atenção do usuário e podem acabar interrompendo uma tarefa que
está sendo executada no momento, dado que a atenção humana é um recurso finito \cite{simon1971designing}.

Interrupções de motoristas já são um problema a ser considerado. Nos Estados Unidos, 10\% dos acidentes com morte, em 2014, tiveram relação
com algum tipo de distração e 13\% deste tipo de distrações estão relacionadas com o uso de celulares e \textit{smartphones}
\cite{distracted2014}. Diante disto, prevenir este tipo de distração relacionada a dispositivos móveis torna-se uma necessidade.

A utilização de dispositivos móveis durante a direção de veículos, apesar de ser proibido no Brasil, é liberado em alguns
outros países, como Estados Unidos (em alguns estados como Flórida e Colorado \cite{cellphoneuse, distracteddriving} e Suécia \cite{swedendrive}.

Ao identificar padrões nos dados contextuais que indiquem momentos oportunos para interromper motoristas, é possível
construir uma solução que os notifique apropriadamente a depender do contexto que eles se encontram. Este trabalho propõe
o desenvolvimento de um sistema sensível ao contexto que detecte momentos oportunos e inoportunos para notificação de motoristas, utilizando
apenas sensores de smartphone. Este módulo será incorporado ao aplicativo Meu Possante, que serve como um auxiliar de mecânica automotiva.

O sistema supracitado será responsável por gerenciar o momento de entrega das notificações geradas pelo Meu Possante ao dispositivo. Ao
incluí-lo no código há uma melhoria na segurança do usuário, já que as notificações não serão entregues em momentos indevidos e
colocando em risco a vida dos mesmos quando estiverem dirigindo.

Neste trabalho a seguinte metodologia foi utilizada: inicialmente investigamos momentos oportunos e inoportunos para interrupção de motoristas
a partir da leitura de artigos da área. Tendo em mente os momentos escolhidos para este trabalho (curvas, mudança de faixa, veículo parado e
velocidade baixa e constante), buscamos limiares na literatura que nos permitam identificá-los utilizando apenas sensores de smartphone.
Com estes dados em mãos, um módulo foi desenvolvido em Android nativo e incorporado em um aplicativo experimental. Por fim, elaboramos um
experimento para avaliar a solução proposta.

O experimento foi feito com três motoristas em situações reais de direção, para verificar se o sistema identificava aproriadamente os momentos
oportunos e inoportunos escolhidos para notificação de motoristas. Os resultados indicaram que o sistema possui uma acurácia de 88\%.

A principal contribuição deste trabalho é a construção de um sistema que identifica momentos oportunos e inoportunos para interrupção de motoristas
usando apenas sensores de smartphone. Possíveis trabalhos futuros são a investigação e identificação de mais momentos, e transformar o sistema
existente em uma biblioteca para que possa ser utilizado em qualquer aplicativo Android.

Os próximos capítulos estão estruturados da seguinte forma: O Capítulo \ref{revisao-lit} discorre sobre os conceitos teóricos importantes para o
entendimento deste trabalho e apresenta os trabalhos relacionados. O Capítulo \ref{proposta} apresenta a proposta deste trabalho, um sistema
sensível ao contexto que detecta momentos oportunos e inoportunos para notificação de motoristas. O Capítulo \ref{estudo-experimental}
apresenta o experimento de avaliação realizado e os resultados obtidos. Por fim, no Capítulo \ref{conclusao} são apresentadas as considerações finais.
