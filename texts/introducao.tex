\xchapter{Introdução}{}
\label{introducao}

Dirigir um veículo pode ser considerado uma tarefa complexa, já que envolve diversas pequenas ações coordenadas e
que dependem tanto de fatores externos quanto internos. Um motorista deve saber adaptar-se a diferentes situações,
agir apropriadamente perante cada uma delas e tomar medidas rápidas em caso de emergência.

Neste contexto, a interrupção inoportuna é um perigo que pode ser fatal, pois demanda atenção do motorista em um momento
onde a distração pode levar a um acidente. Um fato que corrobora isto é o de que pessoas levam até 27\% mais tempo para
concluir uma tarefa quando são interrompidas e tendem a cometer o dobro de erros \cite{bailey2006need}. Exemplos de
interrupções que podem ocorrer para motoristas são barulhos repentinos, conversas com passageiros e notificações em
seu celular.

Outra informação que sustenta esta hipótese é a de que ao demandar a atenção dos motoristas durante a execução de
tarefas complexas sua distração também aumenta, pois a quantidade de informação que um motorista pode assimilar
depende da complexidade de sua tarefa no momento \cite{schneegass2013data}.

Um dos meios pelos quais os motoristas recebem interrupções são via notificações em dispositivos móveis.
Diariamente, pessoas que utilizam qualquer tipo de dispositivo móvel recebem dezenas de notificações
\cite{pielot2014situ}. Tais notificações demandam a atenção do usuário e podem acabar interrompendo uma tarefa que
está sendo executada no momento, dado que a atenção humana é um recurso finito \cite{simon1971designing}.

Interrupções de motoristas já são um problema a ser considerado. Nos Estados Unidos, 10\% dos acidentes com morte, em 2014, tiveram relação
com algum tipo de distração e 13\% deste tipo de distrações estão relacionadas com o uso de celulares e \textit{smartphones}
\cite{distracted2014}. Diante disto, prevenir este tipo de distração relacionada a dispositivos móveis torna-se uma necessidade.

A utilização de dispositivos móveis durante a direção de veículos, apesar de ser proibido no Brasil, é liberado em alguns
outros países, como Estados Unidos (em alguns estados como Flórida e Colorado \cite{cellphoneuse, distracteddriving} e Suécia \cite{swedendrive}.

Ao identificar padrões nos dados contextuais que indiquem momentos oportunos para interromper motoristas, é possível
construir uma solução que os notifique apropriadamente a depender do contexto que eles se encontram. Este trabalho propõe
o desenvolvimento de um módulo que detecte momentos oportunos e inoportunos para notificação de motoristas, utilizando
apenas sensores de smartphone. Este módulo será incorporado ao aplicativo Meu Possante, que serve como um auxiliar de mecânica automotiva.

O módulo supracitado será responsável por gerenciar o momento de entrega de certas notificações ao dispositivo. Ao incluí-lo no código,
o aplicativo terá a garantia de que suas notificações não estão sendo entregues em momentos indevidos e colocando em risco a vida dos
motoristas que usam smartphone.

Neste trabalho a seguinte metodologia foi utilizada: inicialmente investigamos momentos oportunos para interrupção de motoristas a partir
da leitura de artigos da área. Tendo em mente os momentos escolhidos para este trabalho (curvas e mudança de faixa), buscamos limiares na
literatura que nos permitam identificá-los utilizando apenas sensores de smartphone. Com estes dados em mãos, um módulo foi desenvolvido
em Android nativo e incorporado em um aplicativo experimental. Por fim, elaboramos um experimento para avaliar a solução proposta.

O experimento foi feito usando lorem ipsum dolor sit amet, consectetur adipisicing elit, sed do eiusmod tempor incididunt ut labore et dolore
magna aliqua. Ut enim ad minim veniam, quis nostrud exercitation ullamco laboris nisi ut aliquip ex ea commodo consequat. Duis aute irure dolor
in reprehenderit in voluptate velit esse cillum dolore eu fugiat nulla pariatur. Excepteur sint occaecat cupidatat non proident, sunt in
culpa qui officia deserunt mollit anim id est laborum.

O restante deste trabalho está estruturado da seguinte maneira: O Capítulo 1 Lorem ipsum dolor sit amet, consectetur adipisicing elit, sed do
eiusmod tempor incididunt ut labore et dolore magna aliqua. Ut enim ad minim veniam, quis nostrud exercitation ullamco laboris nisi ut aliquip
ex ea commodo consequat. Duis aute irure dolor in reprehenderit in voluptate velit esse cillum dolore eu fugiat nulla pariatur. Excepteur sint
occaecat cupidatat non proident, sunt in culpa qui officia deserunt mollit anim id est laborum.
