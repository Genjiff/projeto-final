\xchapter{Proposta}{}
\label{proposta}
Os desafios supracitados motivaram o desenvolvimento de uma biblioteca que detecte momentos oportunos
e inoportunos para detecção de motoristas, utilizando apenas sensores de smartphone.
Este capítulo aborda os principais aspectos da proposta para este projeto. As seguintes seções são apresentadas:
Metodologia, Resultados Esperados e Cronograma.

\section{Metodologia}
\label{metodologia}

\begin{enumerate}
  \item Serão identificados momentos oportunos para interrupção de motoristas a partir da leitura de artigos
relacionados;
  \item Serão escolhidos os sensores de smartphone que a aplicação utilizará para coletar dados contextuais, para
que seja possível identificar corretamente os momentos indicados no passo 1;
  \item Um aplicativo Android será criado para coleta de dados dos sensores escolhidos no passo 2;
  \item Será feita uma experimentação em situações reais de direção veicular, de forma a relacionar os valores indicados
nos sensores com momentos oportunos para interrupção;
  \item Será feita a investigação e escolha do melhor algoritmo de aprendizado de máquina para classificação dos dados,
baseado nos dados preliminares obtidos no passo anterior. Esta investigação será feita através da ferramenta Weka
\cite{hall2009weka};
  \item Será feito um estudo de caso para a avaliação dos resultados utilizando o modelo obtido no passo anterior;
\end{enumerate}

\section{Resultados Esperados}
\label{resultados}

Ao final deste trabalho espera-se ter uma biblioteca que faça a predição automática de momentos oportunos para
interrupção de motoristas. No futuro, este mecanismo poderá ser implementado na forma de uma biblioteca, podendo assim
ser utilizada por qualquer aplicativo do sistema operacional Android.

\section{Cronograma}
\label{cronograma}

\begin{table}[h]
\centering
\resizebox{\textwidth}{!}{
\begin{tabular}{|r|c|c|c|c|c|c|c|c|c|}
\hline
\multicolumn{1}{|c|}{\multirow{2}{*}{Atividade}}                                                             & \multicolumn{2}{c|}{Dezembro} & \multicolumn{2}{c|}{Janeiro} & \multicolumn{2}{c|}{Fevereiro} & \multicolumn{2}{c|}{Março} & Abril     \\ \cline{2-10}
\multicolumn{1}{|c|}{}                                                                                       & 1ª quinz.     & 2ª quinz.     & 1ª quinz.     & 2ª quinz.    & 1ª quinz.      & 2ª quinz.     & 1ª quinz.    & 2ª quinz.   & 1ª quinz. \\ \hline
\begin{tabular}[c]{@{}r@{}}Pesquisa de momentos oportunos\\ para interrupção de motoristas\end{tabular}      & X             &               &               &              &                &               &              &             &           \\ \hline
\begin{tabular}[c]{@{}r@{}}Pesquisa de sensores utilizados para\\ obtenção de dados contextuais\end{tabular} & X             &               &               &              &                &               &              &             &           \\ \hline
\begin{tabular}[c]{@{}r@{}}Elaboração da arquitetura do\\ aplicativo\end{tabular}                            &               & X             & X             &              &                &               &              &             &           \\ \hline
\begin{tabular}[c]{@{}r@{}}Testes com sensores que serão\\ utilizados\end{tabular}                           &               &               &               & X            &                &               &              &             &           \\ \hline
Desenvolvimento do aplicativo                                                                                &               &               & X             & X            & X              & X             &              &             &           \\ \hline
Experimentação                                                                                               &               &               &               &              &                & X             &              &             &           \\ \hline
\begin{tabular}[c]{@{}r@{}}Escolha do algoritmo de\\ aprendizado de máquina\end{tabular}                     &               &               &               &              &                &               & X            &             &           \\ \hline
Estudo de caso                                                                                               &               &               &               &              &                &               & X            &             &           \\ \hline
Escrita da monografia                                                                                        &               & X             & X             & X            & X              & X             & X            & X           &           \\ \hline
Entrega do projeto                                                                                           &               &               &               &              &                &               &              & X           &           \\ \hline
Defesa do PF II                                                                                              &               &               &               &              &                &               &              &             & X         \\ \hline
\end{tabular}
}
\end{table}
