\xchapter{Proposta}{}
\label{proposta}

Neste trabalho é proposto um sistema de notificação sensível ao contexto com identificação de momentos oportunos e
inoportunos para interrupção de motoristas, usando apenas sensores de smartphone. As seções deste capítulo estão
estruturadas da seguinte forma: A Seção \ref{meupossante} apresenta o Meu Possante, um aplicativo para motoristas
que serviu como motivação para este trabalho. A Seção \ref{sec-momentos-oportunos-inoportunos} discorre sobre
os momentos oportunos e inoportunos escolhidos para a identificação no sistema. A Seção \ref{sec-arquitetura-solucao}
apresenta a arquitetura da solução proposta. A Seção \ref{sec-implementacao} mostra detalhes da implementação
de cada módulo que compõe a solução.

\section{Meu Possante}{}
\label{meupossante}
Mecânica automotiva é um assunto extremamente complexo. Cada um dos inúmeros componentes
do automóvel tem seu próprio ciclo de vida e devem ser revisados e trocados em seu próprio
tempo. Além disso, certas condições de uso podem diminuir a vida útil de algumas peças,
tornando mais frequente a necessidade de revisão.

A falta de conhecimento destes fatos pode levar ao dono de um automóvel negligenciar
as manutenções no tempo correto, causando desde transtornos que poderiam ser facilmente
evitados a acidentes ocasionados por falhas mecânicas.

Em 2016, mais de 2 milhões de automóveis foram vendidos no Brasil
\cite{fenabrave}. Dentre este número, é seguro assumir que poucos de seus
proprietários são especialistas em mecânica e possuem dúvidas sobre o
funcionamento das peças do seu veículo, além de não saber exatamente em qual
momento deve-se trocar cada uma de suas peças. Estas informações geralmente
estão no manual do veículo, mas é inviável para uma pessoa leiga memorizar
todas estas informações.

Após a compra de um automóvel, a sua manutenção é de inteira responsabilidade
do proprietário. Todas as operações de manutenção, especificadas pelo fabricante,
devem ser realizadas dentro dos intervalos apropriados \cite{manualhyundai}.
Proporcionar manutenção apropriada para o veículo, não somente reduz os custos
operacionais, mas também ajuda a impedir mau funcionamento devido a negligência,
caso que geralmente não é coberto por garantia \cite{manualonix}.

Para executar a manutenção apropriadamente, o proprietário precisa estar
sempre atento ao momento correto da troca das peças, que muda de acordo
com as condições de uso de um carro. Acompanhar estas diferentes variáveis pode
ser difícil para pessoas comuns.

O uso de um aplicativo para celular pode ser uma grande ajuda na decisão de quando
é necessário a revisão e troca de peças, alertando o usuário visualmente quando alguma
manutenção está próxima. Neste sentido surge o Meu Possante, um aplicativo que monitora
o estado atual do veículo e avisa o momento em que as manutenções das peças serão
necessárias.

O Meu Possante possui um objetivo simples: Auxiliar o motorista a identificar quais
peças de seu carro precisarão de manutenção em um futuro próximo. O aplicativo nasceu
como trabalho final da disciplina de Desenvolvimento de Aplicações para Dispositivos
Móveis e está em processo refinamento para publicação na Google Play Store.

A tela principal do aplicativo mostra todos os componentes do veículo que são monitorados,
como pode ser visto na Figura \ref{meu-possante-tela-principal}. Ao clicar em um componente,
a tela do mesmo é mostrada, com informações como a frequência de troca e quilometragem
atual.

\begin{figure}[h]
\centering
\includegraphics[width=0.3\textwidth]{images/meu-possante-tela-principal.png}
\caption{Tela principal do Meu Possante}
\label{meu-possante-tela-principal}
\end{figure}

O aplicativo utiliza informações do GPS do dispositivo para calcular a distância percorrida
pelo motorista enquanto está dirigindo. No momento em que detecta a iminência de manutenção
de alguma das peças do veículo, uma notificação é enviada para o dispositivo do usuário.
Esta notificação leva à página da peça correspondente no aplicativo, onde pode ser lido
mais detalhes sobre o estado atual, como a quilometragem e qual a quilometragem da
próxima troca, como pode ser visto na Figura \ref{meu-possante-tela-componente}.

\begin{figure}[h]
\centering
\includegraphics[width=0.3\textwidth]{images/meu-possante-tela-componente.png}
\caption{Tela que indica a necessidade de revisão de um componente}
\label{meu-possante-tela-componente}
\end{figure}

Um exemplo do caso supracitado é o fluido de freio, cuja troca recomendada é a cada
10.000 quilômetros rodados ou 1 ano, o que ocorrer primeiro. Já a troca da correia
dentada deve ser feita a cada 50.000 quilômetros, sem limite de tempo definido. Os
outros componentes funcionam da mesma forma, cada qual com sua quilometragem e
tempo de troca específicos.

Na versão atual, o motorista precisa indicar que está dirigindo através de uma tela
específica. Ao indicar que está dirigindo, o aplicativo começa a monitorar a localização
do usuário e seu deslocamento. Durante a corrida, os dados de quilometragem são atualizados
no banco de dados e na iminência da manutenção de uma peça, uma notificação é enviada para o
usuário.

\subsection{Arquitetura}
\label{meupossante-app}

O Meu Possante foi pensado de forma a ter uma arquitetura simples e eficiente. Na figura
\ref{meu-possante-arquitetura} é possível ver o fluxo de dados através dos módulos, começando pela
entrada de dados do GPS e chegando ao final com a atualização destes dados em um banco de dados
e o eventual envio de uma notificação. As setas na imagem indicam o sentido do fluxo de dados.

\begin{figure}[h]
  \centering
  \includegraphics[width=0.7\textwidth]{images/arquitetura-meu-possante.png}
  \caption{Arquitetura do Meu Possante}
  \label{meu-possante-arquitetura}
\end{figure}

Os módulos utilizados são os seguintes:

\begin{enumerate}
  \item \textbf{Location Service:} Responsável pela configuração e gerenciamento do sensor de localização (GPS).
    O monitoramento do sensor é feito através de um serviço que continua rodando no background,
    mesmo quando o usuário não está com a aplicação aberta.
  \item \textbf{Service Handler:} Responsável pela inicialização e checagem de dados dos serviços
    que estão rodando na aplicação. Este módulo também é responsável por atualizar as informações
    de quilometragem no banco de dados e notificar o usuário, caso detecte a iminência de
    manutenção de alguma peça.
\end{enumerate}

A solução proposta no presente trabalho almeja modificar esta arquitetura para que o aplicativo não notifique motoristas
em momentos inoportunos, evitando colocá-los em potenciais riscos. A próxima seção apresenta os momentos oportunos
e inoportunos escolhidos para serem identificados.

\section{Momentos oportunos e inoportunos escolhidos}
\label{sec-momentos-oportunos-inoportunos}

Para a escolha dos momentos oportunos e inoportunos que a proposta iria identificar, foi feita uma prospecção dos artigos
existentes nessa área. Nesta etapa dois artigos se mostraram promissores: O de \citeonline{kim2015sensors} e o
de \citeonline{monk2004recovering}. Ambos trabalhos citam momentos oportunos e inoportunos para detecção da interruptibilidade
de motoristas. Os momentos escolhidos estão exibidos na tabela \ref{tabela-momentos-oportunos-inoportunos}.

\begin{table}[h]
\centering
\caption{Momentos oportunos e inoportunos escolhidos para identificação na proposta}
\label{tabela-momentos-oportunos-inoportunos}
\begin{tabular}{|c|c|}
\hline
\textbf{Momentos Oportunos}                                                & \textbf{Momentos Inoportunos} \\ \hline
Veículo parado                                                             & Curva                         \\ \hline
\begin{tabular}[c]{@{}c@{}}Veículo com velocidade\\ constante\end{tabular} & Mudança de faixa              \\ \hline
\end{tabular}
\end{table}

As próximas subseções explicam com detalhes o porquê das escolhas destes momentos.

\subsection{Momentos oportunos}
\label{subsec-momentos-oportunos}

\citeonline{kim2015sensors} desenvolveram um classificador utilizando aprendizagem de máquina para detectar a interruptibilidade
de motoristas em um dado momento. O resultado do trabalho mostrou que interrupções têm menos impacto em um motorista quando ele
não está com nenhuma das mãos no volante ou quando está interagindo com um periférico (controlador do ar condicionado, por exemplo).

Além disso, pode-se inferir a partir do trabalho de \citeonline{kim2015sensors} alguns outros momentos que são oportunos para
interromper motoristas, como por exemplo quando o carro está parado (todas as vezes em que o motorista não tinha nenhuma das
mãos no volante, a velocidade do carro era menor do que 3 km/h), quando a velocidade do veículo é baixa (durante a interação com
periféricos a velocidade do carro reduzia para cerca de 29,5 km/h) e quando a velocidade é constante.

Com base nestes achados, foi decidido que os momentos oportunos a serem detectados neste trabalho são:

\begin{itemize}
  \item Veículo parado;
  \item Veículo com velocidade menor do que 29,5 km/h e constante (sem aceleração ou desaceleração).
\end{itemize}

\subsection{Momentos inoportunos}
\label{subsec-momentos-inoportunos}

\citeonline{monk2004recovering} estudaram o efeito das interrupções e as distrações que elas causam nos motoristas.
Eles concluem em seu estudo que interrupções durante uma tarefa ou sub-tarefa trazem problemas, e cita curvas, mudanças
de faixa e entrada em uma rodovia como exemplos de sub-tarefas que o motorista executa. Logo, não é indicado notificar
um motorista durante estas atividades.

Com base nestas afirmações, os seguintes momentos inoportunos foram escolhidos para serem detectados no presente trabalho:

\begin{itemize}
  \item Curva;
  \item Mudança de faixa.
\end{itemize}

\section{Arquitetura}
\label{sec-arquitetura-solucao}
Segundo \citeonline{garlan1993introduction}, a arquitetura de um software é a coleção de seus componentes computacionais - ou simplesmente
componentes - junto com a descrição das interações entre estes componentes - os conectores. Sendo assim, nesta seção
vamos mostrar os principais elementos da solução proposta e como eles interagem entre si.

O módulo funcionará em conjunto com o aplicativo Meu Possante, um aplicativo para sistemas Android e cuja arquitetura pode ser vista
na Seção \ref{meupossante-app}. A identificação dos momentos oportunos e inoportunos é feita usando apenas sensores de smartphone,
sendo que os momentos oportunos são medidos utilizando o sensor de GPS, enquanto os inoportunos usam o giroscópio.

O fluxo de dados da aplicação pode ser visto na Figura \ref{arquitetura-meu-possante-com-modulo}. O fluxo começa
na coleta de dados do GPS e giroscópio, e termina na decisão de notificação ou não do usuário. As setas na
imagem representam o sentido do fluxo de dados.

\begin{figure}[h]
\centering
\includegraphics[width=0.85\textwidth]{images/arquitetura-meu-possante-com-modulo.png}
\caption{Visão geral do Meu Possante após a implementação do módulo de identificação.}
\label{arquitetura-meu-possante-com-modulo}
\end{figure}

Os módulos representados e suas funções são os seguintes:

\begin{enumerate}
  \item \textbf{Location Service:} Responsável pela coleta e monitoramento dos dados do GPS, incluindo deslocamento, velocidade e
  aceleração do dispositivo.
  \item \textbf{Gyroscope Service:} Responsável pela coleta e monitoramento dos dados do giroscópio. Este serviço detecta mudanças
  nos valores do sensor e aplica o algoritmo de detecção de curvas e mudanças de faixa.
  \item \textbf{Service Handler:} Responsável pela inicialização e checagem de dados dos serviços que estão rodando na aplicação.
  Este módulo também é responsável por atualizar as informações de quilometragem no banco de dados e chamar o módulo de notificação
  quando necessário.
  \item \textbf{Notification Handler:} Responsável por consultar os serviços e decidir se deve criar uma notificação ou não naquele
  momento.
\end{enumerate}

Na próxima Seção são dados mais detalhes sobre os principais módulos implementados na proposta, assim como os algoritmos utilizados.

\section{Implementação}
\label{sec-implementacao}

O projeto foi desenvolvido no Android Studio, o ambiente de desenvolvimento integrado (IDE) oficial para
codificação de aplicativos Android. A linguagem utilizada foi o Java, linguagem padrão para o desenvolvimento
de aplicações Android.

As próximas subseções apresentam detalhes sobre a implementação dos principais módulos que compõem o sistema.

\subsection{Location Service}
\label{location-service}

O módulo \textit{Location Service} é o serviço responsável pela obtenção dos dados de GPS da aplicação. Os principais dados calculados
por este serviço são a distância percorrida desde a última leitura, o valor da velocidade e a aceleração. Esta última é necessária
apenas para determinar se a velocidade é constante ou não.

Por ser necessário obter a localização do usuário em intervalos regulares, foi preciso especificar o intervalo de tempo que o
serviço requisita atualizações da localização. Esta configuração tem impacto direto na autonomia de bateria do dispositivo,
pois quanto mais frequente é a requisição de dados do GPS, mais energia é gasta pelo dispositivo. O intervalo de tempo escolhido
foi de 5000ms (5 segundos), um intervalo razoavelmente frequente e que preserva a autonomia de bateria.

A distância percorrida é um dado necessário para o Meu Possante calcular a quilometragem percorrida pelo veículo até o momento e
sinalizar as trocas de peças quando necessário. Já os valores de velocidade e aceleração são necessários para determinar se um
momento é oportuno ou inoportuno para notificar um motorista. A relação entre os momentos e os dados que são usados para
identificá-los estão relacionados na tabela \ref{tabela-momentos-dados}.

\begin{table}[h]
\centering
\caption{Relação entre momentos oportunos e dados necessários para identificá-los}
\label{tabela-momentos-dados}
\begin{tabular}{|c|c|}
\hline
\textbf{Momento oportuno}                     & \textbf{Dado do serviço utilizado} \\ \hline
Veículo parado                                & Velocidade                         \\ \hline
Velocidade constante e menor do que 29,5 km/h & Velocidade e aceleração            \\ \hline
\end{tabular}
\end{table}

Mais informações sobre os referidos momentos podem ser lidas na Seção \ref{sec-momentos-oportunos-inoportunos}.

Dois métodos importantes deste módulo são o \lstinline[basicstyle=\ttfamily\color{black}]|isAccelerating()|, que retorna um valor
booleano que julga se o veículo está acelerando ou não, e o método \lstinline[basicstyle=\ttfamily\color{black}]|getLastSpeed()|, que
retorna a última velocidade medida do veículo. Estes dois métodos são utilizados pelo módulo \textit{Notification Handler} para decidir se
a notificação pode ser disparada ou não.

Os valores de velocidade e distância percorrida são dados por métodos padrões da API Location, a API padrão para se trabalhar
com dados de geo-referenciamento em Android. Entretanto, não há um método padrão nesta API para medir a aceleração, e por este
motivo foi necessária a criação de um método para calculá-la. O método foi estruturado da seguinte forma:

\begin{enumerate}
  \item As últimas 5 leituras de velocidade são sempre guardadas;
  \item Verifica-se a variação de velocidade entre elas;
  \item Caso a variação entre uma leitura e sua subsequente seja menor do que 5 km/h mais o desvio padrão das leituras em questão,
  considera-se que a velocidade é constante naquele momento.
\end{enumerate}

Caso a aceleração seja constante e a velocidade seja menor que 29,5 km/h, significa que o momento é oportuno para notificar o motorista.

\subsection{Gyroscope Service}
\label{gyroscope-service}

Este serviço é o responsável pela obtenção dos dados de giroscópio. Este é o módulo mais complexo desta proposta, por ser o responsável por
detectar se o veículo está fazendo uma curva ou mudança de faixa usando apenas os dados do giroscópio. O algoritmo usado para identificar
estes dois momentos inoportunos foi desenvolvido por \citeonline{chen2015invisible} e adaptado para este trabalho.

\citeonline{chen2015invisible} relatam em seu trabalho que quando um carro muda de direção (ex: ao mudar de faixa, fazer uma curva ou passar
por rodovias sinuosas), o eixo Z do giroscópio pode ser usado para representar a velocidade angular do veículo para aquela mudança de direção.
Eles também perceberam que o padrão das leituras deste eixo se repete para determinadas atividades, como curvas e mudanças de faixa, como pode
ser visto na Figura \ref{leituras-giroscopio}.

\begin{figure}[h]
\centering
\includegraphics[width=1\textwidth]{images/leituras-giroscopio.png}
\caption{Leituras do giroscópio quando o veículo faz uma curva à direita/esquerda ou mudança de faixa à direita/esquerda \cite{chen2015invisible}}
\label{leituras-giroscopio}
\end{figure}

Os gráficos (a) e (b) na Figura \ref{leituras-giroscopio} mostram a alteração nas leituras do giroscópio durante uma curva; esta alteração pode ser positiva ou negativa,
a depender do sentido da curva. Os gráficos (c) e (d) demonstram que a mudança de faixa caracteriza-se por duas alterações sequenciais na leitura do giroscópio,
sendo a segunda no sentido inverso da primeira.

Observando estes padrões, \citeonline{chen2015invisible} desenvolveram um conjunto de regras que determina se a alteração nas leituras do eixo Z do
giroscópio pode ser considerada um indicador de curva ou mudança de faixa. As regras são as seguintes:

\begin{enumerate}
  \item Todas as leituras durante a alteração devem ser maiores que $\delta_{s}$;
  \item O maior valor registrado durante a alteração deve ser maior do que $\delta_{h}$;
  \item A duração de uma alteração não deve ser menor do que $T_{BUMP}$.
\end{enumerate}

Através de testes e experimentos, os valores ideais encontrados por \citeonline{chen2015invisible} para as variáveis acima foram de $\delta_{s}$ = 0.05 rad/s,
$\delta_{h}$ = 0.07 rad/s e $T_{BUMP}$ = 1.5 segundos.

Considerando estas regras para o que é considerada uma alteração válida, \citeonline{chen2015invisible} desenvolveram um algoritmo baseado em mudança de estados
que tenta identificar quando o veículo está fazendo uma curva ou mudança de faixa. O algoritmo está representado na Figura \ref{algoritmo-giroscopio}.

\begin{figure}[h]
  \centering
  \includegraphics[width=0.45\textwidth]{images/algoritmo-giroscopio.png}
  \caption{Algoritmo de detecção de curvas e mudanças de faixa \cite{chen2015invisible}}
  \label{algoritmo-giroscopio}
\end{figure}

Os três estados do algoritmo são os seguintes:

\begin{itemize}
  \item \textbf{No-Bump:} Neste estado, nenhuma alteração foi detectada até o momento. Caso o valor do giroscópio passe de $\delta_{s}$, a alteração começa a ser
  monitorada e o estado passa a ser \textit{One Bump};
  \item \textbf{One-Bump:} Este estado começa quando o valor do giroscópio passa de $\delta_{s}$ e termina quando o valor volta a ser menor do que $\delta_{s}$.
  Caso as três regras definidas para uma alteração ser válida forem satisfeitas, o estado passa a ser \textit{Waiting-for-Bump}, caso contrário o estado volta
  a ser \textit{No-Bump}.
  \item \textbf{Waiting-for-Bump:} Este estado sucede o de \textit{One-Bump} e monitora as leituras do valor do giroscópio por um tempo de 3 segundos. Caso
  uma outra alteração comece durante esse intervalo de tempo, a alteração começa a ser monitorada. Caso ela seja válida, significa que duas alterações válidas
  ocorreram seguidamente, o que caracteriza uma mudança de faixa. Caso contrário, significa que apenas uma alteração válida ocorreu, o que caracteriza uma curva.
\end{itemize}

A solução proposta pelo presente trabalho utiliza o algoritmo de \citeonline{chen2015invisible} para detectar os momentos inoportunos de notificação: curva
e mudança de faixa. O método \lstinline[basicstyle=\ttfamily\color{black}]|isAbleToNotify()| é o responsável por informar ao módulo \textit{Notification Handler}
se o momento atual é oportuno ou não.
