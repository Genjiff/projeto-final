\xchapter{Interrupção}
\label{interrupcao}
Controle cognitivo é o processo que permite que o comportamento do indíviduo e seu processamento de informação variem de momento a
momento, ao invés de se manterem rígidos e inflexíveis. O controle cognitivo pode ser expresso de várias maneiras, uma das quais
envolve a seleção do próximo pensamento para se focar, quando há múltiplas opções e quando distrações podem intervir.

Um exemplo do fato acima é uma conversação, que geralmente segue uma linha coerente. Se uma interrupção ocorre (um dos celulares dos
interlocutores começa a tocar, por exemplo) esta linha pode ser perdida, levando-os a se perguntar "Onde nós estávamos?" \cite{altmann2014momentary}.

Segundo \citeonline{ferreira2004novo}, interrupção é aquilo que faz parar uma ação ou um estado; o ato de cortar a continuidade de
algo. \citeonline{mcfarlane1997interruption} define interrupção humana como o processo de coordenar mudanças abruptas em uma atividade.

A interrupção durante a execução de uma tarefa pode ter vários efeitos adversos. \citeonline{lewin1927untersuchungen} diz
que pessoas lembram melhor dos detalhes de tarefas que não foram interrompidas. \citeonline{zijlstra1999temporal} concluem que
pessoas cometem mais erros em tarefas após uma interrupção. \citeonline{gillie1989makes} afirmam que as pessoas executam tarefas
mais vagarosamente após uma interrupção, se comparado com a performance pré-interrupção. Diante destas evidências, pode-se
afirmar que interrupções durante uma tarefa são bastante nocivas para a execução da mesma.

\section{Interrupção de Motoristas}
\label{interrupcao-motoristas}

Trabalhos anteriores estudaram os efeitos da execução de tarefas concorrentes com a direção. \citeonline{monk2004recovering} citam
que há diversos efeitos adversos ao executar tarefas cognitivas complexas durante a direção, como atraso na resposta a
acontecimentos repentinos, desatenção a informações sinalizadas, diminuição do controle do veículo, estreitamento do campo
de visão e mudanças de comportamento de frenagem e direção.

Além disso, vários problemas na execução de uma tarefa após uma interrupção, como os citados na seção \ref{interrupcao}, afetam o
motorista durante a direção de um veículo:

\begin{itemize}
  \item Ao não lembrar de detalhes do que estava fazendo antes da interrupção, o motorista pode esquecer de informações
  apontadas pela sinalização de trânsito;
  \item Ao cometer erros após uma interrupção, o motorista põe em risco a si mesmo e a seus pares, podendo causar acidentes
  de trânsito;
  \item Ao reagir mais vagarosamente após uma interrupção, o motorista fica vulnerável a ameaças externas que exijam de sua
  capacidade reativa;
\end{itemize}

Estima-se que o uso de celular durante o ato de dirigir um veículo aumenta o risco de acidentes em 38\% \cite{laberge2001wireless}.
Em paralelo a este fato, \citeonline{stothart2015attentional} afirmam que apenas o ato de receber uma notificação, mesmo que ela não seja
atendida, distrai o motorista tanto quanto receber uma ligação no celular ou responder uma mensagem de texto. Ante este fato, é
necessário estudar as notificações e sua capacidade de interromper tarefas.
