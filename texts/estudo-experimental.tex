\xchapter{Estudo Experimental}{}
\label{estudo-experimental}

Este capítulo apresenta o processo de avaliação utilizado para medir a precisão do
sistema de identificação de momentos oportunos e inoportunos proposto. As seções deste
capítulo estão organizadas da seguinte forma: A seção x lorem ipsum.

\section{Metodologia}
\label{metodologia}

O desempenho da proposta foi avaliado a partir de um experimento executado durante
situações reais de direção em um percurso em algumas ruas da cidade de Salvador -
Bahia.

O smartphone escolhido para realizar o experimento foi um OnePlus One com as seguintes características
\cite{oneplusone}:

\begin{itemize}
  \item Sistema Operacional: Cyanogen OS 13.1.2 (Android Marshmallow 6.0.1);
  \item CPU: Quad-Core 2.5 GHz Qualcomm Snapdragon 801;
  \item Memória RAM: 3 GB;
  \item Capacidade de armazenamento: 64 GB.
\end{itemize}

Para a realização do experimento foi criado no aplicativo Meu Possante uma tela com 4 botões, que serviu
para a identificação manual de cada momento. A tela pode ser vista na figura \ref{tela-experimento}.

\begin{figure}[h]
\centering
\includegraphics[width=0.3\textwidth]{images/tela-experimento.png}
\caption{Tela utilizada para auxiliar na identificação manual do momentos oportunos e inoportunos.}
\label{tela-experimento}
\end{figure}

Durante o experimento, ao passar por uma curva, por exemplo, o avaliador pressionava o botão
referente à curva e informações do estado atual dos sensores eram guardadas em uma tabela no banco de dados.
Além disso, cada vez que o sistema detectava um momento oportuno ou inoportuno, um log sobre o
momento era guardado em outra tabela. Este log foi importante para a verificação dos falsos positivos.

O trajeto percorrido pelos participantes tinha 2,6 km de distância e era composto de 5 curvas à direita,
3 curvas à esquerda, 2 lugares com mudança de faixa e 3 semáforos. O experimento foi realizado com 3
voluntários, todos dirigindo em seu próprio carro. O percurso está apresentado na figura \ref{percurso} e os
indivíduos participantes do experimento estão listados na tabela \ref{participantes}.

\begin{figure}[H]
\centering
\includegraphics[width=0.53\textwidth]{images/percurso.png}
\caption{Percurso realizado no experimento de avaliação.}
\label{percurso}
\end{figure}

\begin{table}[h]
\centering
\caption{Indivíduos participantes do experimento}
\label{participantes}
\begin{tabular}{|c|c|c|c|}
\hline
\textbf{Indivíduo} & \textbf{Idade} & \textbf{Sexo} & \textbf{Carro utilizado} \\ \hline
Indivíduo 1        & 22             & Masculino     & Fiat Bravo               \\ \hline
Indivíduo 2        & 24             & Masculino     & Chevrolet Corsa          \\ \hline
Indivíduo 3        & 26             & Masculino     & Honda Fit                \\ \hline
\end{tabular}
\end{table}

Antes do início do experimento os participantes foram instruídos a dirigirem por todo o trajeto com
velocidade constante de 25 km/h. Esta instrução juntamente com os elemento do trajeto permitiu que todos
os momentos, oportunos e inoportunos, fossem avaliados durante o experimento.
