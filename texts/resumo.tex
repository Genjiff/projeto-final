A notificação inoportuna de motoristas é um problema real que pode causar distrações e interrupções no trânsito e
consequentemente acidentes. Pessoas normais levam até 27\% mais tempo para concluir uma tarefa quando são interrompidas
e tendem a cometer o dobro de erros. Um dos meios pelos quais motoristas recebem interrupções são via notificações
em dispositivos móveis, que podem chegar à ordem de centenas por dia. Apesar disso, mesmo sabendo de sua capacidade
interruptiva, notificações são valorizadas pelos usuários e fazem parte do próprio hábito de uso dos smartphones.
Sendo assim, como diminuir o potencial interruptivo das notificações sem eliminá-las completamente?

Para tentar mitigar este problema, o presente trabalho propõe a construção de um sistema de notificação sensível ao
contexto com identificação de momentos oportunos e inoportunos para notificação de motoristas. O sistema proposto usa
apenas sensores de smartphone (giroscópio e GPS) para alcançar este objetivo, e infere se o motorista pode ser
interrompido em um dado momento. Um experimento de avaliação foi conduzido com motoristas em situações reais de direção
para verificar se o sistema conseguia identificar momentos oportunos e inoportunos para notificar apropriadamente. Ao
final do experimento apurou-se que o sistema obteve uma acurácia de 88\%.

\begin{keywords}
Interrupção, notificação, motoristas, contexto.
\end{keywords}
